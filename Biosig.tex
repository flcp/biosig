%% Das Layout des Dokumentes wird festgelegt. Hier A4 mit einer 10er Schrift. Typ: Report 
\documentclass[a4paper,10pt,oneside]{article}

\usepackage[ngerman]{babel}
\usepackage[utf8]{inputenc}
\usepackage[T1]{fontenc}
\usepackage{amsmath}

%%Wenn ein Absatz nicht eingerückt werden soll
\setlength{\parindent}{0em}
 
%%Abstand zwischen Absätzen
\setlength{\parskip}{2.0ex plus 1.0ex minus 0.5ex}
 
%% das war der Vorspann, in dem die Formatierungsregeln festgelegt werden
 
%%Nun kommt das Dokument:
 
%%Anfang
\begin{document}

\tableofcontents
 
\section{fickerboy}
Hier steht der Inhalt.
hören und sehen du ficker
äöü

\section{11 - HMM (Hidden Markov Models)}
	\begin{itemize}
		\item Modellieren Sequenz von Datenpunkten
		\item benötigen zugrundeliegendes state modeling
		\item oft zusammen mit GMMs verwendet
	\end{itemize}
	
\subsection{Sequenzmodellierung und State-Modelierung}
	\begin{itemize}
		\item Sequenzmodellierung ist in typischer Signalverarbeitungskette letzte Schritt nach Datenverarbeitung und State Modeling 
		\item Klassifikation und Sequenzmodellierung eng miteinander verbunden
		\item 
	\end{itemize}

\subsection{Dynamic Time Warping}
	\begin{itemize}
		\item einfaches Verfahren zum Vergleich von Sequenzen
		\item Algorithmen in der HMM-Modellierung sehr ähnlich zu DTW
		\item Wir haben: Aufnahmen von Sprachsignalen - Trainingsdaten (Beispielaufnahmen mit bekanntem Inhalt) + Testdaten (Aufnahmen mit unbekanntem Inhalt)
		\item Ziel: Wir wollen die Distanz einer unbekannten Sequenz und einer Beispielsequenz berechnen
		\item Frame für Frame-Vergleich Probleme: Signale sind unterschiedlich lang + Anfang und Ende der Äußerung nicht bekannt
		\item Faggot-Lösung: Lineares Alignment - für fast alle Zwecke aber viel zu unflexibel
		\item Killer-Lösung: DTW
				\begin{itemize}	
					\item basiert auf Prinzip des dynamischen Programmierens (DP) bzw. der minimalen Editierdistanz
					\item Pfade durch eine Matrix von möglichen Zuordnungen berechnet
					\item Ergebnis: Distanzmaß zwischen den beiden Äußerungen
				\end{itemize}
	
		\item Ziel: Finde Distanz zwischen den beiden Äußerungen (je niedriger desto besser)
		\item Problem: Alle Pfade müssen betrachtet werden um den Besten zu finden
		\item Lösung:
				\begin{itemize}
					\item Berechne für jede Zeit $t$ 
				\end{itemize}
 		\end{itemize}
 
%%Ende
\end{document}